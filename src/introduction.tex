\chapter{Introduction}\label{ch:introduction}
The introduction is one of the most important chapters of your thesis.
Most readers will start reading your thesis here and it mainly serves two purposes.

Firstly, the introduction should carefully introduce the reader to your specific research area and your particular thesis topic.
It may be helpful to think about the discipline of computer science as map, where each research area represented as its own country.
Your particular research topic might only be a very small (but not insignificant) town in one of these countries.
Instead of getting right to the point, you slowly "zoom in" and, while doing so, describe the land surrounding it.
This will help the reader to gradually get to know your research area and understand the thesis's topic more thoroughly.
Be carefully not to get lost in specifics and keep it concise.
Notably, the introduction should be readable not only for experts in your field but for computer scientists in general.

Secondly, the introduction should answer the reader's questions as to why your thesis is important, what the problem and current state of art is and why solving this problem will be a valuable contribution to the scientific community.
Additionally, the introduction should already list your specific scientific and technical contributions and optionally include a short description of the structure of your thesis.
